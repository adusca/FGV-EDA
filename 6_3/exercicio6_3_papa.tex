\documentclass[11pt]{article}
\usepackage[utf8]{inputenc}
\usepackage[T1]{fontenc}
\usepackage{fixltx2e}
\usepackage{graphicx}
\usepackage{longtable}
\usepackage{float}
\usepackage{wrapfig}
\usepackage{rotating}
\usepackage[normalem]{ulem}
\usepackage{amsmath}
\usepackage{textcomp}
\usepackage{marvosym}
\usepackage{wasysym}
\usepackage{amssymb}
\usepackage{hyperref}
\tolerance=1000
\usepackage{minted}
\usemintedstyle{perldoc}
\usepackage{tikz}
\hyphenation{es-co-lhi-dos}
%\usepackage[table]{xcolor}
%\usetikzlibrary{decorations.markings}
%\tikzstyle{vertex}=[circle, draw, inner sep=0pt, minimum size=7pt]
%\newcommand{\vertex}{\node[vertex]}
\author{Alice Duarte Scarpa, Bruno Lucian Costa}
\date{2015-06-23}
\title{Exercício 6.3 (Papadimitriou)}
\hypersetup{
  pdfkeywords={},
  pdfsubject={},
  pdfcreator={Emacs 24.4.1 (Org mode 8.2.10)}}
\begin{document}

\maketitle

\section{Enunciado}
\label{sec-1}

O Yuckdonald's está considerando abrir uma cadeia de restaurantes em
Quaint Valley Highway (QVH). Os $n$ locais possíveis estão em uma
linha reta, e as distâncias desses locais até o começo da QVH são, em
milhas e em ordem crescente, $m_1, m_2, \ldots, m_n$. As restrições
são as seguintes:
\\

\noindent
\begin{itemize}


\item Em cada local, o Yuckdonald's pode abrir no máximo um restaurante. O lucro esperado ao abrir um restaurante no local 
$i$ é $p_i$, onde $p_i > 0$ e $i = 1, 2, \ldots, n$.
\item Quaisquer dois restaurantes devem estar a pelo menos $k$ milhas de distância, onde $k$ é um inteiro positivo.

\end{itemize}

Dê um algoritmo eficiente para computar o maior lucro total
esperado, sujeito às restrições acima.



\section{Introdução}
\label{sec-2}

Com este exercicio vamos abordar uma técnica chamada de programação dinâmica que tem como caracteristica que 
a solução ótima pode ser calculada de soluções de subproblemas.

Antes porém, vai ser apresentado uma solução utilizando um algoritmo guloso. 


\section{Soluções para o problema}
\label{sec-3}

\subsection{Algoritmo gulosos}

\label{sec-3-3}

Esse algoritmo recebe duas lista de tamanho $n$, uma com as distancias dos locais até o ponto inicial e a outra com os respectivos lucros esperados, e um inteiro $k$ que é a distancias em milhas desejada.  
E começamos nosso algoritmo saindo do ponto inicial, em direção ao fim da QVH 

\begin{minted}[]{python}

def restaurante(distancias,k,lucros):
    
    lmax = max(distancias) # Valor maximo das distancias
    rest = []
    lucro = []
    lucro.append(0) # iniciando lucro de 0 loja, como 0
    j = 0 # posicao da loja
    for i in range(lmax):# percorrendo toda a QVH 
        if j in distancias: #Quando local estiver disponivel
            rest.append(distancias[distancias.index(j)]) # guarda posicao escolhida
            lucro.append(lucros[distancias.index(j)]) # guarda lucro escolhido
            j=j+k # pula k milhas a frente
        else:
            j += 1 # anda 1 milha a frente
    return sum(lucro) #retorna soma dos lucros escolhidos

\end{minted}

Vamos reproducir um exemplo no qual esse algoritmo não retonar o valor máximo possivel e vamos tentar entender.

\begin{minted}[]{python}

Chamada da funcao:
restaurante([3, 8, 9, 15],3,[5, 6, 10, 8])

Resultado:
19
 
\end{minted}

O resultado obtido utilizando desse algoritmo não foi o resultado ótimo pois nesse exemplo é fácil perceber que o valor máximo 
que se pode ter respeitando as restrições é de 23, no qual a escolha seria feita pelos locais $[3, 9, 15]$, porém o algoritmo guloso 
está instruido sempre a escolher o primeiro local vago respeitando as restrições, ou seja nesse exemplo ele escolhe os locais $[3, 8, 15]$ 
totalizando o lucro de 19 que é inferior ao valor ótimo. O algoritmo guloso funcionaria bem para o caso que todos os locais tem o mesmo
lucro esperado. 

Vamos resolver esse problema com utilizando um algoritmo baseado no paradigma de programação dinâmica.

\subsection{Algoritmo utilizando programação dinâmica}

Esse técnica de programação utiliza as soluções dos sub-problemas para calcular a solução do problema.

Então vamos definir o nosso sub-problema: Suponha $L(i)$ como o lucro máximo que podemos obter com os locais de $1 \text{ até } i$ e que
$L(0)=0$ então nosso algoritmo deve seguir a seguinte regra: $$ L(i) = \max{(L(i-1),p_i + L(i_{dispo}))} ,$$  
onde $i_{dispo}$ é o maior $j$ tal que $m_j \leq m_i - k$, ou seja o primeiro local antes de $i$ que esteja a pelo menos $k$ milhas de distancia. 


Esse algoritmo recebe duas lista de tamanho $n$, uma com as distancias dos locais até o ponto inicial e a outra com os respectivos lucros esperados, e um inteiro $k$ que é a distancias em milhas desejada.

\begin{minted}[]{python}

def restaurante(distancias,k,pay):
    lucro = [0 for a in range(len(pay)+1)] #iniciando vetor de zeros de tamanho n+1 
    for i in range(len(distancias)):
        m_novo = distancias[i]-k # restricao de nova posicao disponivel
        i_est=[b for b in distancias if b <= m_novo] #Calculando os m_j
        if len(i_est) > 0:
            i_est=i_est[-1] # pegando maior m_j
            d_est= distancias.index(i_est) # pegando o indice i do m_j
            d_novo=lucro[d_est+1] #calculando L(i_dispo)
        else:
            d_novo=0
        lucro[i+1]=max(lucro[i],pay[i]+d_novo) #calculo lucro acumulado da posicao i

    return lucro[-1] #retorna o maior lucro calculado


\end{minted}

Vamos reproducir o mesmo exemplo que utilizamos no algoritmo guloso e vamos perceber que o valor que retornamos é o valor máximo.

\begin{minted}[]{python}

Chamada da funcao:
restaurante([3, 8, 9, 15],3,[5, 6, 10, 8])

Resultado:
23
 
\end{minted}

Isso se deve ao que a programação dinâmica utiliza os valores de lucros já calculado e guardados na memória para calcular os valores ótimos futuros. 




\section{Complexidade}
\label{sec-5}


A solução obtida pelo algoritmo guloso possui complexidade linear, porém não é ótima, como podemos perceber no exemplo.

A solução obtida pelo algoritmo utlizando programação dinâmica é a solução ótima e possui complexidade linear.



\end{document}