\documentclass[11pt]{article}
\usepackage[utf8]{inputenc}
\usepackage[T1]{fontenc}
\usepackage{fixltx2e}
\usepackage{graphicx}
\usepackage{longtable}
\usepackage{float}
\usepackage{wrapfig}
\usepackage{rotating}
\usepackage[normalem]{ulem}
\usepackage{amsmath}
\usepackage{textcomp}
\usepackage{marvosym}
\usepackage{wasysym}
\usepackage{amssymb}
\usepackage{hyperref}
\tolerance=1000
\usepackage{minted}
\usemintedstyle{perldoc}
\usepackage{tikz}
\hyphenation{es-co-lhi-dos}
%\usepackage[table]{xcolor}
%\usetikzlibrary{decorations.markings}
%\tikzstyle{vertex}=[circle, draw, inner sep=0pt, minimum size=7pt]
%\newcommand{\vertex}{\node[vertex]}
\author{Alice Duarte Scarpa, Bruno Lucian Costa}
\date{2015-06-23}
\title{Exercício 4.5 (Tardos)}
\hypersetup{
  pdfkeywords={},
  pdfsubject={},
  pdfcreator={Emacs 24.4.1 (Org mode 8.2.10)}}
\begin{document}

\maketitle

\section{Enunciado}
\label{sec-1}

Vamos considerar uma rua campestre longa e quieta, 
com casas espalhadas bem esparsamente ao longo da mesma. 
(Podemos imaginar a rua como um grande segmento de reta, 
com um extremo leste e um extremo oeste.) 
Além disso, vamos assumir que, apesar do ambiente bucólico, 
os residentes de todas essas casas são ávidos usuários de telefonia celular.
\\

\noindent
Você quer colocar estações-base de celulares em certos pontos da rodovia, 
de modo que toda casa esteja a no máximo quatro milhas de uma das estações-base. 
Dê um algoritmo eficiente para alcançar esta meta, usando o menor número possível de bases.



\section{Introdução}
\label{sec-2}

Com este exercicio vamos abordar uma técnica chamada de algoritmos gulosos
sempre realizando a escolha que parece ser a melhor no momento, fazendo uma escolha ótima local, 
com intuito de que esta escolha leve até a solução ótima global.

Antes porém, vai ser apresentado soluções utilizando algoritmos ``naive'' e um força bruta. 


\section{Soluções para o problema}
\label{sec-3}


\subsection{Naive algoritmo}
\label{sec-3-1}

Esta primeira solução para o problema é uma das mais simples possiveis de se pensar quando confrontamos o problema.
O problema diz que temos que colocar uma antena a no máximo 4 milhas de distancias, nesse algoritmo fizemos a solução 
baseado apenas nessa ideia, então com ele vamos colocar uma antena a cada 4 milhas de distancia até que a casa mais distante 
esteja coberta pela nossas antenas.


\begin{minted}[]{python}
 
def antena(lista):
    lmax = max(lista) # Valor maximo presente na lista de distancias
    ant = []
    j = 0
    for i in range(lmax): # Coloca uma antena a cada 4 milhas
        if j >= lmax: #Ver se a antena tem posicao maior que maximo da lista
            return ant
        j += 4
        ant.append(j)
          
\end{minted}

Esse algoritmo bem simples nos retorna uma solução correta para o problema, mas ele ainda nos faz colocar muitas antenas 
de forma desnecessárias como podemos ver no exemplo a seguir.

\begin{minted}[]{python}

Chamada da funcao:
print antena([3, 16, 11, 18, 5, 17, 24, 29, 1, 301])

Resultado:
 [4, 8, 12, 16, 20, 24, 28, 32, 36, 40, 44, 48, 52, 56, 60, 64, 68, 
 72, 76, 80, 84, 88, 92, 96, 100, 104, 108, 112, 116, 120, 124, 128, 
 132, 136, 140, 144, 148, 152, 156, 160, 164, 168, 172, 176, 180, 184, 
 188, 192, 196, 200, 204, 208, 212, 216, 220, 224, 228, 232, 236, 240, 
 244, 248, 252, 256, 260, 264, 268, 272, 276, 280, 284, 288, 292, 296, 300, 304]

 
\end{minted}


Outro algoritmo ``naive'' que tem uma solução melhor do que anterior será apresentado a seguir.

\begin{minted}[]{python}
 def antena2(lista):
    ant=[]
    for i in lista: #Pecorre toda a lista
        ant.append(lista[lista.index(i)]) #Para cada item da lista coloca uma antena

    ant.sort()
    return ant
\end{minted}

Neste algoritmo a ideia seria colocar uma antena para cada casa o que resolveria nosso problema.
Vamos reproducir o mesmo exemplo feito com o algoritmo anterior para vermos a diferença entre as soluções.

\begin{minted}[]{python}

Chamada da funcao:
print antena2([3, 16, 11, 18, 5, 17, 24, 29, 1, 301])

Resultado:
[1, 3, 5, 11, 16, 17, 18, 24, 29, 301]

 
\end{minted}

Já conseguimos perceber uma diferença muito grande entre as soluções.

Esses dois algoritmos até agora apresentados não nos retorna a melhor solução, os proximos algoritmos tentaremos conseguir a
solução ótima para resolução deste problema.

\subsection{Força Bruta}

\label{sec-3-2}

Esse algoritmo de força bem simples escolhe um ponto qualquer dentro dessa rua para coloca uma antena, depois disso ele percorre 
toda a lista para ver se tem alguma casa que é coberta por essa antena, se tiver retiramos essa casa da lista e efetuamos esse 
procedimento até que todas as casas tenham sido cobertas.

\begin{minted}[]{python}
import math, numpy

def antena(lista):
  lmax = max(lista)# Valor maximo presente na lista de distancias
  ant = []
  
  while lista != []: # Realizar procedimento ate todas as casas cobertas
      torre = numpy.random.randint(1, lmax) #fixando uma torre em um ponto qualquer
      for j in lista: #Passando toda a lista
          if j >= torre-4 and j <= torre+4: # Verifica se tem casa esta coberta
              lista.remove(j) # remove a casa coberta
              ant.append(torre) # adciona a torre a lista
  
  ant = list(set(ant)) # Remove as torres colocadas em duplicatas
  ant.sort() #Ordena as torres

  return ant
 
 
\end{minted}

Vamos reproducir o mesmo exemplo feito com o algoritmo anterior para vermos a diferença entre as soluções.

\begin{minted}[]{python}

Chamada da funcao:
print antena2([3, 16, 11, 18, 5, 17, 24, 29, 1, 301])

Resultado:
[4, 7, 10, 18, 20, 25, 28, 299]
 
\end{minted}


A solução do algoritmo para esse problema pode até ser a ótima eventualmente mas em suma ele demorar mais a conseguir um resposta
para o problema devido a sua escolha aleatoria do local a colocar a antena. 

Em outras palavras esse algoritmo trabalha muito parecido com o jogo de batalha naval, ele escolhe aleatoriamente uma antena para
colocar porém algumas vezes pode escolher em local vazio gerando retrabalho o algoritmo.

\subsection{Algoritmo gulosos}

\label{sec-3-3}

Esse algoritmo recebe uma lista com as distancias das casas até o ponto inicial. 
E começamos nosso algoritmo saindo do ponto inicial, ao oeste, em direção ao leste até que primeira casa esteja 4 milhas a oeste 
colocamos uma antena neste local e retiramos da lista todas as casas cobertas por essa antena. Depois continuamos com esse processo
até todas as casas serem retiradas da lista.

\begin{minted}[]{python}

def antena(lista):
    lmax = max(lista)# Valor maximo presente na lista de distancias
    ant = []
    j = 0
    for i in range(lmax):
        if j >= lmax:
            if j - ant[-1] <= 4:
                return ant
        if j in lista:
            j += 4
            ant.append(j)
            j += 4
        else:
            j += 1
    return ant
 
\end{minted}

Vamos reproducir o mesmo exemplo feito com o algoritmo anterior para vermos a diferença entre as soluções.

\begin{minted}[]{python}

Chamada da funcao:
print antena([3, 16, 11, 18, 5, 17, 24, 29, 1, 301])

Resultado:
[5, 15, 28, 305]
 
\end{minted}

Esse algoritmo sempre nos retorna a solução ótima e vamos mostrar isso a seguir.

Suponha $S = \{ s_1, \ldots s_k \}$ sendo a solução com as posições das antenas 
que o nosso algoritmo retornou e $T = \{ t_1, \ldots t_m \}$ sendo a solução ótima com as posições 
das antenas ordenadas de forma crescente. Queremos mostrar que $k=m$.

Vamos mostrar nosso algoritmo $S$ ``stay ahead'' da solução $T$. Ou seja, $s_i \geq t_i$.
Para $i = 1$ essa afirmação é verdade, já que vamos ao leste o máximo possivel antes de colocar a antena.
Iremos assumir também é verdade para $ i \geq 1$, ou seja, $\{ s_1 \ldots s_i \}$ cobre as mesmas casas que $\{ t_1 \ldots t_i \}$, 
então se adicionarmos $t_{i+1}$ para $\{ s_1 \ldots s_i \}$, não deixa nenhuma casa entre $s_i$ e $t_{i+1}$ descobertas.
Mas no passo $(i+1)$ do algoritmo guloso é escolhido o $s_{i+1}$ para ser o maior possivel com a condição cobrir as casas entre $s_i$ e 
$s_{i+1}$ e então $s_{i+1} > t_{i+1}$ o que prova o que queriamos.

Então, se $k>m$, a solução $\{ s_1 \ldots s_m \}$ falha ao cobrir todas as casas, mas $s_m \geq t_m$ logo $\{ t_1 \ldots t_m \} = T$
também falha ao cobrir todas as casas. O que é uma contradição, pois assumimos que $T$ era uma solução ótima para o problema.




\section{Complexidade}
\label{sec-5}

TODO: calcular a complexidade do algoritmo



\end{document}