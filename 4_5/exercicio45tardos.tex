\documentclass[11pt]{article}
\usepackage[utf8]{inputenc}
\usepackage[T1]{fontenc}
\usepackage{fixltx2e}
\usepackage{graphicx}
\usepackage{longtable}
\usepackage{float}
\usepackage{wrapfig}
\usepackage{rotating}
\usepackage[normalem]{ulem}
\usepackage{amsmath}
\usepackage{textcomp}
\usepackage{marvosym}
\usepackage{wasysym}
\usepackage{amssymb}
\usepackage{hyperref}
\tolerance=1000
\usepackage{minted}
\usemintedstyle{perldoc}
\usepackage{tikz}
\hyphenation{es-co-lhi-dos}
%\usepackage[table]{xcolor}
%\usetikzlibrary{decorations.markings}
%\tikzstyle{vertex}=[circle, draw, inner sep=0pt, minimum size=7pt]
%\newcommand{\vertex}{\node[vertex]}
\author{Alice Duarte Scarpa, Bruno Lucian Costa}
\date{2015-06-23}
\title{Exercício 4.5 (Tardos)}
\hypersetup{
  pdfkeywords={},
  pdfsubject={},
  pdfcreator={Emacs 24.4.1 (Org mode 8.2.10)}}
\begin{document}

\maketitle

\section{Enunciado}
\label{sec-1}

Vamos considerar uma rua campestre longa e quieta, 
com casas espalhadas bem esparsamente ao longo da mesma. 
(Podemos imaginar a rua como um grande segmento de reta, 
com um extremo leste e um extremo oeste.) 
Além disso, vamos assumir que, apesar do ambiente bucólico, 
os residentes de todas essas casas são ávidos usuários de telefonia celular.
\\

\noindent
Você quer colocar estações-base de celulares em certos pontos da rodovia, 
de modo que toda casa esteja a no máximo quatro milhas de uma das estações-base. 
Dê um algoritmo eficiente para alcançar esta meta, usando o menor número possível de bases.



\section{Introdução}
\label{sec-2}

Com este exercicio vamos abordar uma técnica chamada de algoritmos gulosos
sempre realizando a escolha que parece ser a melhor no momento, fazendo uma escolha ótima local, 
com intuito de que esta escolha leve até a solução ótima global.

Antes porém, vai ser apresentado duas soluções utilizando um algoritmo ``naive'' e um força bruta. 


\section{Soluções para o problema}
\label{sec-3}



\subsection{Naive algoritmo}

Esta primeira solução para o problema é uma das mais simples possiveis de se pensar quando confrontamos o problema.
O problema diz que temos que colocar uma antena a no máximo 4 milhas de distancias, nesse algoritmo fizemos a solução 
baseado apenas nessa ideia, então com ele vamos colocar uma antena a cada 4 milhas de distancia até que a casa mais distante 
esteja coberta pela nossas antenas.


\begin{minted}[]{python}
 
 def antena(lista):
    lmax = max(lista) # Valor maximo presente na lista de distancias
    ant = []
    j = 0
    for i in range(lmax): 
        if j >= lmax:
            if j - ant[-1] <= 4:
                return ant
        j += 4
        ant.append(j)
 
\end{minted}


\subsection{Força Bruta}

\label{sec-3-1}


\begin{minted}[]{python}
import math, numpy

def antena(lista):
  lmax = max(lista)
  ant = []
  
  while lista != []: # Realizar procedimento ate todas as casas cobertas
      torre = numpy.random.randint(1, lmax) #fixando uma torre em um ponto qualquer
      for i in range(len(lista)): #Percorrendo toda a lista
        for i in lista: # 
            if i >= torre-4 and i <= torre+4: # Verifica se tem casa esta coberta
                lista.remove(i) # remove a casa coberta
                ant.append(torre) # adciona a torre a lista
  
  ant = list(set(ant)) # Remove as torres colocadas em duplicatas
  ant.sort() #Ordena as torres

  return ant
 
 
\end{minted}



\section{Implementação}
\label{sec-4}

\subsection{Fluxo válido com demandas não-nulas}
\label{sec-4-2}



\section{Complexidade}
\label{sec-5}

TODO: calcular a complexidade do algoritmo



\end{document}